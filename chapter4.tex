\chapter{Results}
\label{resultschap}

% PENDING: Results
youtube links, charts... from flacco, QP simulations and real robot if posible.

\section{Perception}

The culmination of proximity sensor transformation explain in section in addition to the work done in calibration by OTHERS, will allow to the robot to sense its surroundngs. In simulation, we have shown that proximity sensors shown in the robot body (reference to appear in red) can detect objects in simulation as seen below.

Move the results to this section and show the box in gazebo too.


The update rate achieved in simulation has been 100 Hz. In practical applications the rate is limited by the specific sensor's capabilites, as opposed to the fact that the limiting factor is processing time for depth sensors.

\section{End effector position control with }

Cartesian position control was a necessary step to implement the rest of the avoidance algorithms. We will use this controller as a foundation and to inform the movement of all subsequent controllers.

Without mid-range happens this. Mid-range thing works very well. This works because SIMPLE EXPLANATION OF THE BEHAVIOR. For more detailed explanation of this see section.

if the threshold that decides if you have arrived to the desired point is too small, the control period does not allow a fast enough update to stop at the desired location.

video starting in a bad position and showing how it moves in circles. Explain the content of the video.

\section{Flacco}

Show how the end effector algorithm affects the movement

Local minimum


\section{QP}

Why is the second restriction bad.

Acceleration limits make it better. They haven't implemented it although they do mention it, but we've seen it's important.

\section{HIRO}

List in Caleb's presentation.

Video.
