\chapter{Introduction}
\label{introchap}

\section{Human-Robot Interaction}
We are on the threshold of a new era in robotic technology that will bring breaking improvements into our lives. Simple household robots, such as the vacuuming robot Roomba\textsuperscript{\textregistered} \cite{roomba} or the kitchen robot thermomix\textsuperscript{\textregistered} \cite{thermomix}, are starting to do some of human's the most time consuming tasks. But this is just the beginning and robots will soon be more extended in our daily lives. This robotic revolution we are talking about has the potential to surpass the ubiquity and usefulness of the computer revolution, if only because robots can change not only our virtual world, but the physical world also.

Human-robot interaction (HRI) is a field of study dedicated to understanding, designing, and evaluating robotic systems for use by or with humans. It is one of the main fields of research working towards the revolution presented in the first paragraph and it is focused on expanding robot's abilities to interact with the physical world, integrating robots in human-populated environments. There are a number of ways in which robotics can be impactful to society in the short- mid- term. Some examples:

\begin{itemize}
    \item Hospital robots will check on patients and report their status to nurses, saving time and improving patient outcomes. One current example is the Moxi\textsuperscript{\textregistered} hospital assistant robot \cite{moxi}. It can already accomplish tasks such us gathering supplies and bringing them to patient rooms and delivering lab samples.
    \item Childcare robots will help parents with chores such as assisting at diaper changing or feeding, so that families can spend high-quality time together. iPAL\textsuperscript{\textregistered} \cite{ipal} is a sneak peak of what is to come.
    \item Manufacturing robots will collaborate with people to assemble complex objects on reconfigurable assembly lines, increasing the efficiency and flexibility of factory floors.
\end{itemize}

Nowadays robots in the industry live in restricted access areas and interaction with humans is avoided. The transition of robots from these restricted access areas to human-populated environments comes with a number of problems that need to be overcome. One of the key issues is that human safety is compromised. Traditionally, safety in industrial robots has been achieved isolating the robotic platforms in a cell with safety interlocks to prevent direct interaction. HRI contributions will only be feasible once the coexistence of robots and humans does not come with serious injury risk.

\section{Human senses and sensors in robotics}
In order to achieve the safety requirements of HRI, robots are expected to adapt to the changes in their environment the way humans do. This is why it is important to study how humans interact with their environment.

Interaction between a human being and its environment is accomplished through its senses. One important aspect research in robotics has focused on has been to look for ways to emulate and exploit these human senses. For example, visual sensing has been widely explored in this field. One of the reasons is the development of new low-cost depth sensors such as the Microsoft Kinect \textsuperscript{TM} has allowed to meet the many requirements of vision application at an affordable price. These depth sensors are capable of acquiring 2D images as well as the distances of the objects represented in each pixel to the camera. The usual way this data is used is to reassemble representations of obstacles in a robot-oriented space.

However, other human senses are of great importance for the acquisition of data relevant to the interaction with their surroundings. Touch is one of them. Tactile sensing is the most fundamental sense when contacting the external world and it is the biggest and oldest of the sensorial organs. An experiment that consisted of exploring objects after anesthetizing the hands of a control group demonstrated that the simple task of maintaining a stable grasp of objects becomes surprisingly difficult when sense of touch is suppressed. Their hand movements are inaccurate and unstable \cite{westling1984factors}. In another experiment carried on by astronauts at the International Space Station, sense of touch was proven to be an important indicator of direction and spatial disorientation \cite{van2006touch}. Sense of touch is crucial when experiencing object properties, such as size, shape, texture and temperature. It informs about slip. It is essential to develop awareness of the body and in consequence, to differentiate one’s body from the rest of one’s surroundings. Its absence seriously hinders the interaction of an individual with its environment \cite{dahiya2009tactile}.

Nonetheless, tactile sensing in robotics has not been able to achieve the penetration this evidence suggests it should have had. In traditional industrial robots, this importance has been ignored. Engineers have been able to avoid the issue of developing a artificial sensor that emulates touch by using prior knowledge about the object to be manipulated and the environment. The limitations of this approach are obvious: robots are only capable of working in structured and controlled environments. https://www.sciencedirect.com/science/article/pii/S0921889015001621

The result is that research and technology in artificial tactile sensors is not as well developed as other perception modalities. Tactile sensors in robotic applications are represented by:

\begin{itemize}
    \item\textbf{Pressure sensing arrays.} Pressure sensor matrix that provides information about the location and amount of pressure exerted on a surface.
    \item\textbf{Force-torque sensors.} They give feedback about the forces and torques that are applied on a given point in the 3 geometric axes.
    \item\textbf{Light level change detection}  https://www.wired.com/story/this-clever-robotic-finger-feels-with-light/
\end{itemize}

\section{Whole-body artificial skins}
Current research is focused on developing tactile skins that cover robot end effectors and hands with a number of tactile sensors, in parallel with devising new algorithms that make use of these new sensors for dexterous object manipulation (\cite{tianFeel}, \cite{wilson2020design}).

Even though we know about the importance of human skin covering all the body, whole-body artificial skins have not gained the same attention as end effector coverings. In fact, these types of skins can be of great importance in the transition of robots to operating alongside humans and the new safety issues that arise with it.

Traditional robots have been designed for precision and performance at expenses of safety and the ability to adapt to the surroundings. The two paradigms that try to find a solution to this issue, different but not mutually exclusive, are: \textbf{passive} and \textbf{active compliance}.  Passive compliance aims to achieve safety through physical adaptation to the environment. There are two different strategies to accomplish that. On the one hand, there is the use offlexible parts and mechanisms. On the other hand, limitation of the performance capabilities of the robot, such us speed, can also be enough to achieve safety. Active compliance, in contrast, relies on algorithms and control laws for reactive interaction with the environment.

Both approaches have some important limitations, though. Passive compliance achieves safety at the cost of reducing accuracy, in the case of using flexible parts, and agility when the capabilities of the robot are reduced. As far as active compliance is concerned, sensors that are used are far from being perfect for the control. Some sensors provide either a high amount of low-quality information regarding the task, as in the case of the aforementioned depth cameras or a limited amount of relevant data, as force-torque sensors located at the robot’s joints. In addition, both solutions are currently lacking in flexibility. New problems require completely new designs. These are important enough reasons that explain why none of the solutions have become the established standard in robotics.

Whole-body artificial skins have the potential to solve all the issues present in active and passive compliance. It does so providing high-density relevant information, information that is much more relevant to the task of collision avoidance. At the same time, it can also be designed to be mechanically compliant to the environment.


In addition, the skin can be equipped with heterogeneous sensing so that a greater amount of high-quality information is collected. Tactile data is only useful when contact has been established between the robot and the environment. However, collisions can be actively avoided before establishing undesired contact by choosing suitable pre-impact strategies. That is why  proximity sensing opens up new possibilities for the task of collision avoidance. The addition of proximity sensors allows a compact, external motion capture free system, with highly informative prior-to-contact data. This combined with lighter data, makes proximity sensing the perfect sensing system for the solution of the collision avoidance problem.

\section{Control and collision avoidance}
From the control perspective, using proximity data for the safe movement of a robot might prevent performing the task commanded to a robot, that is, the original trajectory when there were no impediments that prevented the desired movement commanded to perform a certain task. When any undesired object obstructed the desired trajectory of a robot, it is not difficult to see that the performance of that main task could be hindered. Nonetheless, this does not have to be the case for a certain type of robots called redundant robots.

A robot is task redundant if the number of degrees of freedom (DoF) $n$ it has is greater than the degrees of freedom needed for the task $m$. For example, if the specifications of a given task only require positioning the end-effector of a robot at a certain point, i.e. $m = 3$ a robot with $n = 3$ DoF would be able to accomplish the task and one with $n = 4$ DoF would be task redundant. A redundant robot has the advantage of being able to accomplish the task in infinitely many possible ways.

The ability to perform a certain movement of the end effector in infinitely many different ways means that safe motion control does not imply stopping the main task and in fact, the main task should only halt when there is no way to make use of redundancy to avoid a collision while completing it.

\section{Objectives and scope}
% LAST: Objectives. When the results and the work are complete
