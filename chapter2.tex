\chapter{Related Work}
\label{relworkchap}

In this section, we describe the main research areas related to this work and place it in context. We have organized the content in two sections. Section \ref{s:relwork1} contains the previous publications made in the field of whole body artificial skins. Section \ref{s:relwork2} describes the work previously done in collision avoiding robot control.

\section{Whole-body articial skins}
\label{s:relwork1}

One of the first whole-body artificial skin technologies was the iCub skin (\cite{metta2010icub}, \cite{schmitz2011methods}). It made use of an array of capacitive touch sensors for tactile sensing. The arrays were covered by a flexible soft material and later placed on the robotic arms.

A different approach to tactile sensing was made by HEX-o-SKIN (\cite{7793823}) and CellulARSKIN (\cite{5711674}, \cite{mittendorfer2015realizing}). Their skins offer not only pressure sensors, but accelerometers, temperature and proximity sensors too. In spite of these new capabilities, their potential sensor density, that is, their amount of sensors per unit area, was lower. This was due to the fact that the modules were big and difficult to place on high curvature areas.

Another technology that has been utilized for tactile sensing has been the approach followed by Gelsight (\cite{yuan2017gelsight}), Fingervision (\cite{yamaguchi2017implementing}), SoftBubble (\cite{alspach2019soft}). These implementations use optical sensors that are placed under a flexible gel substrate. The benefits of this technology is that the density is very high. However, the limitations come in form of limited scalability for whole-body sensing. The computational power required to process images is enormous. In fact, it is a good technology for end effector tactile sensing and it has been used for dexterous manipulation applications.

In \cite{ding2019proximity}, they present a set of cuffs that can be placed around a robots arms. The cuffs contain proximity and capacitive sensors. Tactile sensing is not covered by this approach. Even their control methods have been inspiring in this project (see \ref{s:relwork2}), we believe that their density is not high enough for the pre-contact collision avoidance task. The rigidity of the cuffs also make them fragile and not compliant.

In the HIRO lab, we are building our own skin upon the characteristics of the different approaches cited until now. From the iCub skin, we focus on its flexibility and high density of sensors. From HEX-o-SKIN and CellulARSKIN we focus on the the heterogeneity of different sensor technologies. In fact, we are adding a human sensor that indicates if the proximity sensor reading comes from a human or not. We are also looking for an approach of individual skin units instead of one large skin patch. This makes the system more modular.

\section{Avoidance control}
\label{s:relwork2}

Collision avoidance can be done at different layers: motion planning and control. Control methods allow robots to reactively respond to sudden changes in the environment.

Most of the collision avoidance control methods are based on a potential fields approach. Artificial repulsive forces associated to obstacles are exerted on the robot . These modify the robot's trajectory in order to avoid obstacles. \cite{flacco2012depth} implements one of this method and it has been implemented as a method for comparison in this work. See section \ref{ss:flacco} for an in depth explanation of the paper.

The formulation of the avoidance control in terms of mathematical optimization has allowed for the introduction of complex behaviours. \cite{nguyen2018compact} suggests a method that prioritizes the avoidance of parts of the human body that are more susceptible of serious damage, such as the head. In \cite{dingcollision} they make use of proximity data to solve the problem. The formulation of the problem via mathematical optimization introduces a new way of forcing repulsive actions, through the addition of restrictions. This last work will be implemented in this work and it is explained in section \ref{ss:optimization}. The final controller we proposed is based on their publication.
